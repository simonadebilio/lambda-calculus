\documentclass[11pt]{article}
\usepackage{amsfonts}
\usepackage{amsmath}
\usepackage{bussproofs}
\usepackage{latexsym}
\usepackage{tipa}
\usepackage{verbatim}

\def\fCenter{{\mbox{\Large$\rightarrow$}}}
\EnableBpAbbreviations


\title{\textbf{Lambda Calculus}}
\author{Luca Violanti\\
		Simona Debilio}
\date{2020-04-01}
\begin{document}

\maketitle

\section*{Church numerals}

\begin{gather*}
F^{0} (M) \equiv M \\
F^{n+1}(M) \equiv F(F^{n}(M)) \\
c_{n} \equiv \lambda fx.f^{n}(x)
\end{gather*}

\section*{Proposition (J.B. Rosser)}
\begin{gather*}
\mathbf{A_{+}} \equiv \lambda xypq.xp(ypq) \\
\mathbf{A_{*}} \equiv \lambda xyz.x(yz) \\
\mathbf{A_{exp}} \equiv \lambda xy.yx
\end{gather*}

\subsection*{Example 1}
\begin{gather*}
\mathbf{A_{+}} c_{0} c_{0} = ((\lambda xypq.xp(ypq)) (\lambda fs.s)) (\lambda gt.t) = \\
= ((\lambda ypq.(\lambda fs.s)p(ypq)) (\lambda gt.t) = \\
= ((\lambda ypq.((\lambda fs.s)p)(ypq)) (\lambda gt.t) = \\
= (\lambda ypq. (\lambda s.s) (ypq)) (\lambda gt.t) = \\
= (\lambda ypq. ypq) (\lambda gt.t) = \\
= \lambda pq. ((\lambda gt.t)p)q = \\
= \lambda pq. (\lambda t.t)q = \\
= \lambda pq. q = c_0
\end{gather*}


\subsection*{Example 2}
\begin{gather*}
\mathbf{A_{+}} c_{1}c_{1} = ((\lambda xypq.xp(ypq)) (\lambda fs.fs)) (\lambda gt.gt) = \\
= (\lambda ypq.((\lambda fs.fs)p)(ypq)) (\lambda gt.gt) = \\
= (\lambda ypq.((\lambda s.ps)(ypq))) (\lambda gt.gt) = \\
= (\lambda ypq.pypq) (\lambda gt.gt) = \\
= (\lambda pq.p (\lambda gt.gt)p)q = \\
= (\lambda pq.p (\lambda t.pt)q = \\
= \lambda pq.ppq 
\end{gather*}


\subsection*{Example 3}
\begin{gather*}
\mathbf{A_{exp}} c_{2}c_{2} = \lambda xy.yx (\lambda fs.f(fs)) (\lambda gt.g(gt)) = \\
= (\lambda y.y (\lambda fs.f(fs))) (\lambda gt.g(gt)) = \\
= \lambda fs.f(fs) (\lambda gt.g(gt)) = \\
= \lambda s. (\lambda gt.g(gt)) ((\lambda gt.g(gt))s) = \\
= \lambda s. (\lambda gt.g(gt)) (\lambda t.s(st)) = \\
= \lambda s. (\lambda t. (\lambda t.s(st)) (s(st)) = \\
= \lambda s. \lambda t. (s(s (s(st)))) = \\
\end{gather*}

\subsection*{Example 3bis}
\begin{gather*}
\mathbf{A_{exp}} c_{2}c_{2} = \lambda xy.yx (\lambda fs.f(fs)) (\lambda gt.g(gt)) = \\
= (\lambda y.y (\lambda fs.f(fs))) (\lambda gt.g(gt)) = \\
= (\lambda gt.g(gt)) (\lambda fs.f(fs)) = \\
= \lambda t. (\lambda fs.f(fs) (\lambda fs.f(fs)t)) = \\
= \lambda t. (\lambda fs.f(fs) (\lambda s.t(ts))) = \\
= \lambda t. (\lambda s. (\lambda s.t(ts) (\lambda s.t(ts)s))) = \\
= \lambda t. (\lambda s. (\lambda s.t(ts) (t(ts))) = \\
= \lambda t. \lambda s.t(t(t(ts)))
\end{gather*}


\section*{Chapter 2 Barendregt}
\subsection*{Exercise 2.1}
\subsubsection*{i) Rewrite according to official syntax:}

\begin{gather*}
M_{1} \equiv y (\lambda x.xy (\lambda zw.yz)) \\
M_{1} \equiv (v ((\lambda v'(v'v)) (\lambda v''(\lambda v'''(vv'')))))
\end{gather*}


\subsubsection*{ii) Rewrite according to the simplified syntax:}

\begin{gather*}
M_{2} \equiv \lambda v' (\lambda v''((((\lambda vv) v')v'')((v''(\lambda v'''(v'v''')))v''))) \\
M_{2} \equiv \lambda xy. ((\lambda z.z) xy)(y(\lambda p.xp)y)
\end{gather*}


\subsection*{$\beta$ reduction}
\[
(\lambda x.M)N = M [x:=N]
\]

\subsection*{Exercise 2.2}
 Prove the following sobstitution lemma. Let $x \not\equiv y$ and $x \not\in FV(L)$. Then:
 
\[
M[x:=N][y:=L]\equiv M [y:=L][x:=N[y:=L]]
\]
by $\beta$ reduction

\begin{gather*}
((\lambda x.M)N)[y:=L]\equiv ((\lambda y.M)L)[x:=N[y:=L]] \\
\lambda y.((\lambda x.M)N)L\equiv ((\lambda y.M)L)
[x:=(\lambda y.N)L] \\
\lambda y.((\lambda x.M)N)L\equiv \lambda x. ((\lambda y.M)L)((\lambda y.N)L) \\
\end{gather*}


\subsection*{Lemma 2.9}
\[
\lambda \vdash (\lambda x_{1} \cdots x_{n}.M)X_{1} \cdots X_{n} = M [x_{1}:=X_{1}] \cdots [x_{n} := X_{n}]
\]

\begin{gather*}
M[x:=N][y:=L]\equiv M [y:=L][x:=N[y:=L]] \\
M[x:=N][y:=L]\equiv M [y:=L][x:=(\lambda y.N)L] \\
((\lambda xy.M)N)L\equiv ((\lambda yx.M)L)(\lambda y.N)L)
\end{gather*}

\subsection*{Exercise 2.5}
Let $\mathbf{B} \equiv \lambda xyz.x(yz)$. Simplify $M \equiv \mathbf{B}XYZ$, that is find a 'simple' term N such that $\lambda \vdash M = N$

\begin{gather*}
M = (\lambda xyz.x(yz))XYZ = \\
= (\lambda yz.X(yz))YZ = \\
= (\lambda z.X(Yz))Z = \\
= X(YZ)
\end{gather*}


\subsection*{Exercise 2.6}
Simplify the following terms

\subsubsection*{i)}
\begin{gather*}
M \equiv (\lambda xyz.zyx)aa(\lambda pq.q) = \\
= (\lambda yz.zya)a(\lambda pq.q) = \\
= (\lambda z.zaa)(\lambda pq.q) = \\
= (\lambda pq.q)aa = \\
= (\lambda q.q)a = \\
= a
\end{gather*}


\clearpage
\subsubsection*{ii)}
\[
M \equiv ((\lambda yz.zy)((\lambda x.xxx)(\lambda x.xxx)))(\lambda w. \mathbf{i)}) =
\]


%M = AB
%	A = CD
%		C = \ y . E
%			E = \ z . H
%				H = MN
%					M = z
%					N = y				
%		D = FG
%			F = \ x . J
%				J = OP
%					O = ST
%						S = x
%						T = x
%					P = x
%				
%			G = \ x . K
%				K = QR
%					Q = UV
%						U = x
%						V = x
%					R = x
%	B = \ w.I
%		I = \ p . L
%			L = p
		
let $D = (\lambda x.xxx)(\lambda x.xxx)$

\begin{gather*}
 = (\lambda yz.zy)D(\lambda w.I) = \\
= (\lambda z.zD)(\lambda w.I) = \\
= (\lambda w. I)D = \\
= I
\end{gather*}


\subsubsection*{iii)}
$\mathbf{S} = \lambda xyz.xz(yz)$

$\mathbf{K} = \lambda xy.x$

\begin{gather*}
M \equiv \mathbf{SKSKSK} = \\
= ((((SK)S)K)S)K = \\
= (((((\lambda xyz.xz(yz))(\lambda xy.x))S)K)S)K = \\
= (((((\lambda yz.(\lambda xy.x)z(yz)))S)K)S)K = \\
= (((((\lambda yz.(\lambda y.z)(yz)))S)K)S)K = \\
= ((((\lambda yz.z)S)K)S)K = \\
= (((\lambda z.z)K)S)K = \\
= (KS)K = \\
= S
\end{gather*}

\subsection*{Exercise 2.7}
\subsubsection*{i) {\boldmath $\mathsf{\lambda \vdash KI = K_*}$}}

\[
(\lambda xy.x)(\lambda z.z) = \lambda pq.q
\]
\[
\lambda yz.z = \lambda pq.q
\]


\subsubsection*{ii) {\boldmath $\mathsf{\lambda \vdash SKK = i)}$}}

\[
(\lambda xyz.xz(yz)) (\lambda xy.x) (\lambda xy.x) = I
\]
\[
(\lambda yz. (\lambda xy.x)z(yz)) (\lambda xy.x) = I
\]
\[
(\lambda yz. (\lambda y.z)(yz)) (\lambda xy.x) = I
\]
\[
(\lambda yz.z) (\lambda xy.x) = I
\]
\[
\lambda z.z = I
\]

\subsection*{Exercise 2.8}

\subsubsection*{i)}
Write down a closed $\lambda$-term $F \in \Lambda$ such that for all $M,N \in \Lambda$

 
\[
FMN = M(NM)N
\]
\[
F = \lambda xy. x(yx)y
\]

\subsubsection*{ii)}
Construct a $\lambda$-term $F \in \Lambda$ such that for all $M, N, L \in \Lambda ^o$

\[
FMNL = N(\lambda x.M)(\lambda yz.yLM)
\]
\[
F = \lambda abc. b(\lambda x.a) (\lambda yz.yca)
\]


\subsection*{Exercise 2.9}
Find closed terms $F$ such that
\subsubsection*{i)}

\[
Fx = x \mathbf{i)}
\]
\[
F = \lambda x. x \mathbf{i)}
\]

\subsubsection*{ii)}
\[
F xy = x \mathbf{i)} y
\]
\[
F = \lambda xy. x \mathbf{i)} y
\]


\subsection*{Exercise 2.10}
Find closed terms $F$ such that

\subsubsection*{i)}
\[
F x = F
\]

\[
F = \lambda x. F
\]


\subsubsection*{ii)}
\[
F x = x F
\]
\[
F = \lambda x. xF
\]

\subsubsection*{iii)}
\[
FIKK = FK
\]
\[
F = \lambda xyz.Fz
\]


\subsection*{Exercise 1.28}
\subsubsection*{a}
\[
(\lambda x.xy) (\lambda u.vuu) =
\]
\[
= (\lambda u.vuu)y =
\]
\[
= vyy
\]


\subsubsection*{b}
\[
(\lambda xy.yx)uv =
\]
\[
(\lambda y.yu)v =
\]
\[
vu =
\]


\subsubsection*{c}
\[
(\lambda x.x(x(yz))x)(\lambda u.uv) =
\]
\[
= ((\lambda u.uv)((\lambda u.uv)(yz))(\lambda u.uv)) =
\]
\[
= ((\lambda u.uv)(yzv)(\lambda u.uv)) =
\]
\[
= (yzvv)(\lambda u.uv)
\]


\subsubsection*{d}
\[
(\lambda x.xxy)(\lambda y.yz) =
\]
\[
= ((\lambda y.yz)(\lambda y.yz)y) =
\]
\[
= ((\lambda y.yz)z)y =
\]
\[
= zzy
\]



\subsubsection*{e}
\[
(\lambda xy.xyy)(\lambda u.uwz) =
\]
\[
= \lambda y.((\lambda u.uwz)yy) =
\]
\[
= \lambda y.ywzy
\]



\subsubsection*{f}
\[
(\lambda xyz.xz(yz))((\lambda xy.yx)u)((\lambda xy.yx)v)w =
\]
\[
= (\lambda xyz.xz(yz)) (\lambda y.yu) (\lambda y.yv) w =
\]
\[
= (\lambda yz.((\lambda y.yu)z)(yz)) (\lambda y.yv) w =
\]
\[
= (\lambda yz.zu(yz)) (\lambda y.yv) w =
\]
\[
= (\lambda z.zu((\lambda y.yv)z)) w =
\]
\[
= (\lambda z.zu(zv)) w =
\]
\[
= wuwv
\]




$\mathbb{LUCA}$\\
$\mathbb{SIMONA FALS}$

\section*{The power of Lambda}

\subsection*{Definition}
\subsubsection*{i)}

\begin{center}$\mathbf{true \equiv K}$ \\
$\mathbf{false \equiv K_*}$\end{center}

\subsubsection*{ii)}
if B is considered as a Boolean, i.e. a term that is either $\mathbf{true}$ or $\mathbf{false}$, then

\begin{center}\emph{if B then P else Q}\end{center}

can be represented by 

\begin{center}\emph{BPQ}\end{center}

\subsection*{Definition (Pairing)}
For $M,N \in \Lambda$ write

\begin{center}$[M, N] \equiv \lambda z.$ if z then M else N \\
($\equiv \lambda z.zMN$)
\end{center}

Then

\begin{center}$[M, N] \mathbf{true} = M$ \\
$[M, N] \mathbf{false} = N$\end{center}

and hence [M, N] can serve as an ordered pair.

\subsection*{Definition}

For each $n \in \mathbb{N}$, the numeral $`n`$ is defined inductively as follows.

\begin{center}
$`0` \equiv I = \lambda z.z$
\end{center}


\begin{center}$`n+1` \equiv [\mathbf{false}, `n`]$ = \\
= $[\mathbf{K_*}, `n`] $ = \\
= $ [\lambda xy. y, `n`]$ = \\
= $\lambda z.z (\lambda xy. y) `n`$
\end{center}


\subsection*{Successor, predecessor, test for zero}

\begin{center}\begin{tabular}{l l l}
$\mathbf{S}^+ `n`$ & = & $`n+1`$ \\
$\mathbf{P}^- `n+1`$ & = & $`n`$ \\
$\mathbf{Zero}`0`$ & = & $\mathbf{true}$ \\
$\mathbf{Zero`n+1`}$ & = & $\mathbf{false}$ \\
\end{tabular}\end{center}

\subsubsection*{Proof}

\begin{center}$\mathbf{Zero} = \lambda x.x \mathbf{true}$ = \\
$ = \lambda x.x K$ = \\
$ = \lambda x.x (\lambda xy.x) $\end{center}


\begin{center}$ \mathbf{Zero} `0` = (\lambda x.x (\lambda xy.x)) (\lambda z.z)$ = \\
= ($\lambda z.z) (\lambda xy.x) $ = \\
= $ \lambda xy.x $ = K = true\end{center}

\begin{center}
$\mathbf{`1`} = \lambda w.w (\lambda xy.y)(\lambda z.z)$ = \\
= $\lambda x.x (\lambda y.y)$ 
\end{center}

\begin{center}
$Zero `1` = (\lambda x.x (\lambda xy.x)) (\lambda z.z (\lambda w.w))$ = \\
= $(\lambda z.z (\lambda w.w))(\lambda xy.x))$ = \\
= $(\lambda xy.x) (\lambda w.w)$ = \\
= $(\lambda y. (\lambda w.w)) $ = \\
= $(\lambda yw.w)) = K_* = false $
\end{center}

\clearpage
\section*{Introduction to Boolean Logic}
https://www.i-programmer.info/babbages-bag/235-logic-logic-everything-is-logic.html?start=2 \\

IF NOT(A$>$0 AND A$<$10) THEN [!(P AND Q)]

\begin{center}\begin{tabular}{l l l}
$A>0$ (P) & $A<10$ (Q) & \\
T & T & F \\
F & T & T \\
T & F & T \\
F & F & T
\end{tabular}\end{center}

IF (A$\le$0 AND A$\ge$10) THEN [(!P AND !Q)]

\begin{center}\begin{tabular}{l l l}
A$\le$0 (!P) & A$\ge$10 (!Q) & \\
T & T & T* \\
F & T & F \\
T & F & F \\
F & F & F
\end{tabular}\end{center}

* A cannot be both A$\le$0 and A$\ge$10

\subsection*{Exercise}
\begin{gather*}
false = K_* = \lambda xy.y\\
[M,N] = \lambda w.wMN\\
\\
'0' = I = \lambda z.z\\
'n+1' = [false, 'n'] = \lambda w.w K_* 'n' =\\
= \lambda w.w (\lambda xy.y)'n' =\\
= \lambda w.w (\lambda y.y)\\
\\
'n' = [false,'n-1'] \quad \forall n>0 = \\
= \lambda w.w K_* 'n-1' = \\
= \lambda w.w (\lambda xy.y) 'n-1' = \\
= \lambda w.w (\lambda y.y)\\
\\
'2' = [false,'2-1'] = [false,'1'] = \\
= [false, [false, '1-1']] = [false, [false, '0']] = \\
= [false, [false, I]] = \dots
\\
\end{gather*}

\subsection*{Execise for Y combinator}
\[
loop = (\lambda x.xx) (\lambda x.xx)
\]
\[
rec \quad f = f (rec \quad f)
\]
\[
rec = \lambda y. y (rec \quad y)
\]
\[
loop \equiv rec (?)
\]
\[
loop \equiv (\lambda y. y (rec \quad y))(?)
\]
\[
(\lambda x.xx)(\lambda x.xx) \equiv (\lambda y. y (rec \quad y))(\lambda z.z)
\]
\[
(\lambda x.xx)(\lambda x.xx) \equiv rec (\lambda z.z) =
\]
\[
= (\lambda z.z)(rec (\lambda z.z)) =
\]
\[
= rec (\lambda z.z)
\]

\begin{gather*}
AB = C \\
A = \lambda b. C
\end{gather*}

\begin{gather*}
fac (0) = 1\\
\forall n> 0 \quad fac (n) = n * fac (n-1) \equiv fac (n+1) = (n+1) * fac(n)\\
\\
fac(n) = if (n == 0) \ then \ 1 \ else \ n * fac(n-1) \\
\\
fac = \lambda m. if (m == 0) \ then \ 1 \ else \ m * fac(m-1) \\
\\
F\ f\ n=(\operatorname {IsZero} \ n)\ 1\ (\operatorname {multiply} \ n\ (f\ (\operatorname {pred} \ n)))\\
fac(n) = F\ f\ n\\
F\ f = fac\\
F = \lambda f. fac \\
F = \lambda f. \lambda m. if (m == 0) \ then \ 1 \ else \ m * f(m-1) \\
F = \lambda f. \lambda m. \ (\operatorname {IsZero} \ n)\ 1\ (\operatorname {multiply} \ n\ (f\ (\operatorname {pred} \ n)))\\
\operatorname {IsZero} = \lambda n.(n==0)\\
\operatorname {multiply} = \lambda ab.(a*b)\\
\operatorname {pred} = \lambda n.(n-1)\\
\\
fac = rec (\lambda f. \lambda m. ?)\\
\\
fac = rec\ (\lambda f. \lambda m. \ (\operatorname {IsZero} \ n)\ 1\ (\operatorname {multiply} \ n\ (f\ (\operatorname {pred} \ n))))\\
\end{gather*}



\begin{gather*}
true = K \\
false = K_* \\
[M,N] \equiv \lambda z.zMN \\
[M,N]true = M \\
(\lambda z.zMN)(\lambda xy.x) = (\lambda xy.x) MN = M \\
[M,N]false = N
\\
\end{gather*}

\begin{gather*}
P^- \equiv \lambda x.x\ \mathbf{false} = \lambda x.x\ (\lambda ab.b) \\
`2` = [\mathbf{false}, [\mathbf{false}, I]] = \\
= \lambda z.(z\ \mathbf{false}\ [\mathbf{false}, I]) = \\
= \lambda z.(z\ \mathbf{false}\ (\lambda p.(p\ \mathbf{false}\ I))) = \\
= \lambda z.(z\ (\lambda ab.b)\ (\lambda p.(p\ (\lambda ab.b)\ (\lambda c.c)))) \\ 
\\
P^- `2`  = (\lambda x.x\ (\lambda ab.b))\ (\lambda z.(z\ (\lambda ab.b)\ (\lambda p.(p\ (\lambda ab.b)\ (\lambda c.c))))) = \\
= (\lambda z.(z\ (\lambda ab.b)\ (\lambda p.(p\ (\lambda ab.b)\ (\lambda c.c)))))\ (\lambda xy.y) = \\
= (\lambda xy.y)\ (\lambda ab.b)\ (\lambda p.(p\ (\lambda ab.b)\ (\lambda c.c))) = \\
= ((\lambda xy.y)\ (\lambda ab.b))\ (\lambda p.(p\ (\lambda ab.b)\ (\lambda c.c))) = \\
= \lambda p.(p\ (\lambda ab.b)\ (\lambda c.c)) = \lambda p. p\ \mathbf{false}\ I = [\mathbf{false}, I] = `1`
\end{gather*}


\begin{gather*}
Add (0, y) = y \\
\forall x > 0,\ Add(x, y) = S^+ (Add (P^- (x), y)) = 1+ Add (x-1, y)\\
\mathbf{Add}\ xy = if\ \mathbf{Zero}\ x\ then\ y\ else\ \mathbf{S^+}(\mathbf{Add} (\mathbf{P^-}x)y)) \\
\mathbf{Add} = \lambda xy.\ if\ \mathbf{Zero}\ x\ then\ y\ else\ \mathbf{S^+}(\mathbf{Add} (\mathbf{P^-}x)y)) \\
\mathbf{Add} = (\lambda axy.\ if\ \mathbf{Zero}\ x\ then\ y\ else\ \mathbf{S^+}(a (\mathbf{P^-}x)y)))\ \mathbf{Add'} \\
\mathbf{Add} \equiv \mathbf{Y} (\lambda axy.\ if\ \mathbf{Zero}\ x\ then\ y\ else\ \mathbf{S^+}(a (\mathbf{P^-}x)y))) \\
\end{gather*}

\clearpage
\subsection*{https://mvanier.livejournal.com/2897.html}

\begin{verbatim}
...
==> ((lambda (n)
         (if (= n 0)
             1
             (* n (factorial0 (- n 1)))))
       2)
       
==> (if (= 2 0)
        1
        (* 2 (factorial0 (- 2 1))))
        
==> (* 2 (factorial0 1))
             
==> (* 2 0) ==> 0

       
\end{verbatim} 


\section*{Exercises Barendregt Chapter 3}
\subsection*{3.1 i)}
Find a $\lambda$ term $\mathbf{Mult}$ such that for all $\emph{m, n} \in \mathbb{N}$

\begin{gather*}
\mathbf{Add} = \lambda xy.\ if\ \mathbf{Zero}\ x\ then\ y\ else\ \mathbf{S^+}(\mathbf{Add} (\mathbf{P^-}x)y)) \\
\mathbf{Mult}\ `n`\ `m` = `n * m` \\
One : \mathbb{N}\ \rightarrow \mathbb{B} \\
\mathbf{One}\ `n` = if\ \mathbf{Zero}\ `n`\ \mathbf{false}\ else\ \mathbf{Zero}(\mathbf{P^-}\ `n`) \\
\\
Mult(1, m) = m \\
\forall n>1,\ Mult(n, m) = Add (m,\ Mult (P^- (n),\ m)) = m + Mult (n-1,\ m) \\
\mathbf{Mult}\ n\ m = if\ \mathbf{One}\ n\ then\ m\ else\ \mathbf{Add}\ m\ (\mathbf{Mult}\ (\mathbf{P^-} n)\ m)
\end{gather*}

\subsubsection*{ 3.1 i Scheme}
\begin{verbatim}
(define Mult
  (lambda (n m)
    (if (= n 1)
        m
        (+ m (Mult (- n 1) m))
    )
  )
)
\end{verbatim}


\subsection*{iBis homework Luca}

\begin{gather*}
Exp (0, m) = 1 \\
\forall n>0,\ Exp(n, m) = Mult (n,\ (Exp (n,\ (P^- (m))))) = n \cdot Exp (n,\ m - 1) \\
\mathbf{Exp}\ n\ m = if\ \mathbf{Zero}\ m\ then\ 1\ else\ \mathbf{Mult}\ n\ (\mathbf{Exp}\ n\ (\mathbf{P^-} m))
\end{gather*}

\begin{verbatim}
(define Exp
  (lambda (n m)
    (if (= m 0)
        1
        (Mult n (Exp n (- m 1)))
    )
  )
)
\end{verbatim}

\subsection*{iTris homework Luca}

\begin{gather*}
Fact 0 = 1 \\
Fact\ n  = n * Fact (n-1)
\\
Fact\ n = if\ \mathbf{Zero}\ n\ then\ 1\ else\ n * (Fact (n-1))
\end{gather*}

\begin{verbatim}
(define Fact
  (lambda (n)
    (if (= n 0)
        1
        (* n (Fact (- n 1)))
    )
  )
)
\end{verbatim}



\begin{gather*}
Fib\ 0 = 0 \\
Fib\ 1 = 1 \\
Fib\ n = Fib (n-1) + Fib (n-2)
\end{gather*}

\begin{flalign*}
& Fib\ nm = if\ \mathbf{Zero}\ n\ then\ 0\ & \\ 
& \hspace{48pt} else\ ( & \\
& \hspace{78pt} if\ \mathbf{One}\ n\ then\ 1 & \\
& \hspace{78pt} else\ \mathbf{Add}\ (\mathbf{Fib}\ (\mathbf{P^-} n))\ (\mathbf{Fib}\ (\mathbf{P^-} (\mathbf{P^-} n))) & \\
& \hspace{74pt} ) &
\end{flalign*}


\begin{verbatim}
(define Fib
  (lambda (n)
    (cond ((= n 0) 0)    
          ((= n 1) 1)
          (else (+ (Fib (- n 1)) (Fib (- n 2))))
    )
  )
)
\end{verbatim}


\subsection*{3.2}
The simple Ackermann function $\varphi$ is defined as follows

\begin{gather*}
\varphi(0, n) = n + 1 \\
\varphi(m + 1, 0) = \varphi (m, 1) \\
\varphi(m + 1, n+1) = \varphi (m, \varphi (m + 1, n)) \\
\end{gather*}
 
Find a $\Lambda$-term F that $\Lambda$-defines $\varphi$.

\begin{flalign*}
& \mathbf{Ack}\ mn = if\ \mathbf{Zero}\ m\ then\ \mathbf{S^+} n & \\
& \hspace{55pt} else\ ( & \\
& \hspace{86pt} if\ \mathbf{Zero}\ n\ then\ \mathbf{Ack}\ (\mathbf{P^-} m)\ 1 & \\
& \hspace{86pt} else\ \mathbf{Ack}\ (\mathbf{P^-} m)\ (\mathbf{Ack}\ m\ (\mathbf{P^-} n)) & \\
& \hspace{80pt} ) &
\end{flalign*}

\begin{verbatim}
(define Ack
  (lambda (m n)
    (cond ((= m 0) (+ n 1))
          ((= n 0) (Ack (- m 1) 1))
          (else (Ack (- m 1) (Ack m (- n 1))))
    )
  )
)
\end{verbatim}

\subsection*{3.3}
The $\lambda$-terms \textbf{Cons}, \textbf{Head} and \textbf{Tail} are defined by

\begin{gather*}
\mathbf{Cons} \equiv \lambda xy. [x,y] \\
\mathbf{Head} \equiv \lambda x. x \mathbf{K} \\
\mathbf{Tail} \equiv \lambda x. x \mathbf{K_*}
\end{gather*}

Find $\lambda$-terms \textbf{nil} and \textbf{Null} such that

\begin{gather*}
\mathbf{Null}\ \mathbf{nil} = \mathbf{true} \\
\mathbf{Null} (\mathbf{Cons}\ xy) = \mathbf{false}
\end{gather*}

Give a representation of lists in $\lambda$-terms and find a $\lambda$-term that defines Append, where Append : Lists x Lists $\rightarrow$ Lists. \\
\\
BNF for lists:
\begin{gather*}
element\ ::=\ \lambda-term \\
list\ ::=\ \mathbf{nil}\ |\ [element,\ list] \\
\mathbf{nil} = (??)
\end{gather*}

\begin{gather*}
\mathbf{nil} := \lambda z.\ \mathbf{true} \\
\mathbf{Null} := \lambda p.p\ (\lambda x. \lambda y. \mathbf{false}) \\
\mathbf{Null}\ \mathbf{nil} = (\lambda p.p (\lambda x. \lambda y.\ \mathbf{false})) (\lambda z.\ \mathbf{true}) = \\
= (\lambda z.\ \mathbf{true}) (\lambda x. \lambda y.\ \mathbf{false}) = \\
= true \\
\\
\mathbf{Cons}\ xy = [x,y] = \lambda z.zxy \\
\mathbf{Null}\ (\mathbf{Cons}\ xy) = (\lambda p.p\ (\lambda x. \lambda y. \mathbf{false}))(\lambda z.zxy) = \\
= (\lambda z.zxy)(\lambda x. \lambda y. \mathbf{false}) = \\
= (\lambda x. \lambda y. \mathbf{false}) xy = \\
= \mathbf{false}
\end{gather*}

\begin{flalign*}
& \mathbf{Head} : \mathbf{List} \rightarrow \mathbf{Elem} &\\
& \mathbf{Tail} : \mathbf{List} \rightarrow \mathbf{List} &\\
& \mathbf{Cons} : \mathbf{Elem} * \mathbf{List} \rightarrow \mathbf{List} &\\
& \mathbf{MakeList} : \mathbf{Elem} * \mathbf{Elem} \rightarrow \mathbf{List} &\\
& \mathbf{MakeList}\ xy = \mathbf{Cons}\ x\ (\mathbf{Cons}\ y\ nil) &\\
\\
& \mathbf{Append} : \mathbf{List} * \mathbf{List} \rightarrow \mathbf{List} &\\
& \mathbf{Append}\ x\ y = if\ \mathbf{Null}\ x\ then\ y\ else\ \mathbf{Cons} (\mathbf{Head}\ (\mathbf{Append}\ (\mathbf{Tail}\ x)\ y)) &
\end{flalign*}

\begin{verbatim}
(define list1 '())
(define list2 '(1 2 3))
(define list3 '(4 5 6))

(define Head car)
(define Tail cdr)

(define Append
  (lambda (x y)
    (cond ((null? x) y)
    ((null? y) x)
    (else (cons (Head x) (Append (Tail x) y)))
    )
  )
)
\end{verbatim}

\subsubsection*{3.4}
Construct $\lambda$ $M_0$ $M_1$ ... such that for all $n$ one has

\begin{gather*}
M_0 = x \\
M_n+1 = (M_n+2) M_n \\
\\
M_n = if\ \mathbf{Zero}\ n\ then\ x\ else\ P^-   
\end{gather*}


\section*{Chapter 4 - Reduction}

\[
\omega \equiv \lambda x. xx
\]
\[
\Omega \equiv \omega \omega
\]

\begin{gather*}
\mathbf{K\ i)}\ \Omega \rightarrow I
(\lambda xy.x)(\lambda x.x) ((\lambda x.xx)(\lambda x.xx))
\end{gather*}

\subsection*{4.24 Touring's fixedpoint combinator}
\begin{gather*}
A \equiv \lambda xy.y(xxy) \\
Tcomb = AA = \lambda y.y(AAy)\\
Tcomb\ F = AAF
\end{gather*}

\clearpage
\section*{Exercises chapter 4}
\subsection*{4.1}
Show $\forall M\ \exists N\ [N\ in\ \beta -nf\ and\ N \mathbf{i)} \rightarrow _\beta M]$ \\
\\
M can be of the form:
\begin{gather*}
v \in V \Rightarrow v \in \Lambda \\
P,Q \in \Lambda \Rightarrow PQ \in \Lambda \\
P \in \Lambda, x \in V \Rightarrow \lambda x.P \in \Lambda
\end{gather*}

Therefore:
\begin{gather*}
N \mathbf{i)} \rightarrow _\beta v \\
(\lambda x.xv)\mathbf{i)} \rightarrow v \\
\\
N \mathbf{i)} \rightarrow _\beta PQ \\
(\lambda a.a PQ)\mathbf{i)} \rightarrow PQ \\
\\
N \mathbf{i)} \rightarrow _\beta \lambda x.P \\
(\lambda k.k (\lambda x.P)) \mathbf{i)} \rightarrow \lambda x.P
\end{gather*}


\subsection*{Exercise 4.8 Barendregt}
Show that the term $M \equiv AAx$ with $A \equiv \lambda axz.z(aax)$ does not have a normal form.

\begin{gather*}
M = (\lambda axz.z(aax))(\lambda axz.z(aax))x = \\
= (\lambda xz.z((\lambda axz.z(aax))(\lambda axz.z(aax))x))x = \\
= \lambda z.z((\lambda axz.z(aax))(\lambda axz.z(aax))x) = \\
= \lambda z.zM
\end{gather*}


\subsection*{Exercise 4.8 Barendregt}
\subsubsection*{i)}
Show that $\mathbf{\lambda} \not \vdash WWW = \mathbf{\omega}_3 \mathbf{\omega}_3$ with $W \equiv \lambda xy.xyy$ and $\mathbf{\omega}_3 \equiv \lambda x.xxx$

\begin{gather*}
WWW = (\lambda xy.xyy)(\lambda xy.xyy)(\lambda xy.xyy) =\\
= (\lambda y.(\lambda xy.xyy)yy)(\lambda xy.xyy) = \\
= (\lambda xy.xyy)(\lambda xy.xyy)(\lambda xy.xyy) \\
\\
\mathbf{\omega}_3 \mathbf{\omega}_3 = (\lambda x.xxx)(\lambda x.xxx) = \\
= (\lambda x.xxx)(\lambda x.xxx)(\lambda x.xxx) = \\
= (\lambda x.xxx)(\lambda x.xxx)(\lambda x.xxx)(\lambda x.xxx) = \dots \\
\end{gather*}

\subsubsection*{ii)}
Show $\mathbf{\lambda} \not \vdash B_x = B_y$ with $B_z \equiv A_z A_z$ and $A_z \equiv \lambda p.ppz$.

\begin{gather*}
B_z = A_z A_z = (\lambda p.ppz)(\lambda p.ppz) = \\
= (\lambda p.ppz)(\lambda p.ppz)z = \\
= ((\lambda p.ppz)(\lambda p.ppz)z)(\lambda p.ppz)z =  \\
= ((\lambda p.ppz)(\lambda p.ppz)z)(\lambda p.ppz)z =
\end{gather*}



\section*{Exercises Chapter 5 - Barendregt}
\subsection*{Exercise 5.1}
\subsubsection*{i)}

Give a derivation of: $\vdash \mathbf{SK} : (\alpha \rightarrow \beta) \rightarrow (\alpha \rightarrow \alpha)$

\begin{gather*}
\mathbf{S}= \lambda xyz. xz (yz) = \\
\lambda x. \lambda y. \lambda z. xz(yz)\\
\mathbf{K}= \lambda xy.x : \sigma \rightarrow \tau \rightarrow \sigma
\end{gather*}

\begin{center}
\AxiomC{$\vdash x : \xi \rightarrow \mu \rightarrow \kappa$}
\AxiomC{$\vdash z : \xi$}
\BIC{$\vdash xz : \mu \rightarrow \kappa$}

\AxiomC{$\vdash y: \xi \rightarrow \mu$}
\AxiomC{$\vdash z: \xi$}
\BIC{$\vdash yz: \mu$}

\BIC{$\vdash xz(yz) : \kappa$}
\AxiomC{$\vdash z : \xi$}

\BIC{$\vdash \lambda z. xz(yz) : \xi \rightarrow \kappa$}
\AxiomC{$\vdash y : \xi \rightarrow \mu$}

\BIC{$\vdash \lambda y. \lambda z. xz(yz) : (\xi \rightarrow \mu) \rightarrow \xi \rightarrow \kappa$}
\DP
\end{center}

\begin{center}
\AxiomC{$\vdash \lambda y. \lambda z. xz(yz) : (\xi \rightarrow \mu) \rightarrow \xi \rightarrow \kappa$}
\AxiomC{$\vdash x : \xi \rightarrow \mu \rightarrow \kappa$}


\BIC{$\vdash \mathbf{S} \equiv  \lambda x. \lambda y. \lambda z. xz(yz) : (\xi \rightarrow \mu \rightarrow \kappa) \rightarrow (\xi \rightarrow \mu) \rightarrow \xi \rightarrow \kappa$}
\DP
\end{center}

\begin{center}
\AxiomC{$\vdash \mathbf{S} \equiv  \lambda x. \lambda y. \lambda z. xz(yz) : (\xi \rightarrow \mu \rightarrow \kappa) \rightarrow (\xi \rightarrow \mu) \rightarrow \xi \rightarrow \kappa$} \AxiomC{$\vdash \mathbf{K} \equiv \lambda ab.a : \alpha \rightarrow \beta \rightarrow \alpha$} 
\BIC{$\vdash \mathbf{SK} : (\alpha \rightarrow \beta) \rightarrow (\alpha \rightarrow \alpha)$}
\DP
\end{center}

With $\xi \equiv \alpha;\ \mu \equiv \beta;\ \kappa \equiv \alpha$

\begin{gather*}
\mathbf{SK} \equiv (\lambda x. \lambda y. \lambda z. xz(yz)) (\lambda ab.a) = \\
= \lambda y. \lambda z. (\lambda ab.a)z(yz) =  \\
= \lambda y. \lambda z. (\lambda b.z) (yz) =  \\
= \lambda y. \lambda z.z = \mathbf{false}????? \\
\end{gather*}

\subsubsection*{ii)}
Give a derivation of: $\vdash \mathbf{KI} : \beta \rightarrow (\alpha \rightarrow \alpha)$

\begin{gather*}
\mathbf{K} = \lambda xy.x : \alpha \rightarrow \beta \rightarrow \alpha \\
\mathbf{I} = \lambda z.z : \gamma \rightarrow \gamma
\end{gather*}

\begin{center}
\AxiomC{$\vdash \mathbf{K} = \lambda xy.x : \delta \rightarrow \epsilon \rightarrow \delta $} \AxiomC{$\vdash \mathbf{I} = \lambda z.z : \gamma \rightarrow \gamma$} 
\BIC{$\vdash \mathbf{KI} : \beta \rightarrow (\alpha \rightarrow \alpha)$}
\DP
\end{center}

With $\gamma \equiv \alpha$ and $\epsilon \equiv \beta$

\subsubsection*{iii)}
Show that $\not \vdash \mathbf{SK} : (\alpha \rightarrow \beta \rightarrow \beta)$
See derivation exercise i (?)

\subsubsection*{iv)}
Find a common $\beta$-reduct of $\mathbf{SK}$ and $\mathbf{KI}$. What is the most general type for this term?


\begin{gather*}
\mathbf{SK} \equiv (\lambda xyz. xz(yz)) (\lambda ab.a) = \\
= \lambda yz. (\lambda ab.a)z(yz) =  \\
= \lambda yz. (\lambda b.z) (yz) =  \\
= \lambda yz.z = \mathbf{false}  \\
\end{gather*}

\begin{gather*}
\mathbf{KI} \equiv (\lambda ab.a)(\lambda c.c) = \\
= \lambda bc.c = \mathbf{false} \\
\end{gather*}

\begin{gather*}
\mathbf{false} \equiv \lambda xy.y : \alpha \rightarrow \beta \rightarrow \beta \\
\end{gather*}

\subsection*{Exercise 5.2}
Show that $\lambda x.xx$ and $\mathbf{KI}(\lambda x.xx)$ have no type in $\lambda \rightarrow$

\begin{center}
\AxiomC{$x : \gamma \rightarrow \beta$} \AxiomC{$ x: \gamma$}
\BIC{$xx : \beta$}

\AxiomC{$ x: \alpha$}
\BIC{$\vdash \lambda x.xx :\ ???$}
\DP
\end{center}

$x$ cannot have simultaneously two different type ``forms": $\gamma \rightarrow \beta$ and $\alpha \equiv \gamma$

\begin{gather*}
\mathbf{KI}(\lambda x.xx) = (\lambda ab.a) (\lambda c.c) (\lambda x.xx) = \\
= (\lambda b.(\lambda c.c))(\lambda x.xx) = \\
= \lambda c.c : \gamma \rightarrow \gamma 
\end{gather*}


\subsection*{Exercise 5.3}
Find the most general types (if they exist) for the following terms.

\subsubsection*{i) $\lambda xy.xyy$}

\begin{center}
\AxiomC{$x: \delta \rightarrow (\delta \rightarrow \epsilon)$}
\AxiomC{$y: \delta$}
\BIC{$xy : \delta \rightarrow \epsilon$}

\AxiomC{$y: \delta$}
\BIC{$xyy : \epsilon$} 

\AxiomC{$y: \delta$}
\BIC{$\lambda y.xyy : \delta \rightarrow \epsilon$}

\AxiomC{$ x: \delta \rightarrow (\delta \rightarrow \epsilon)$}
\BIC{$\vdash \lambda xy.xyy : \delta \rightarrow (\delta \rightarrow \epsilon) \rightarrow (\delta \rightarrow \epsilon)$}
\DP
\end{center}


\subsubsection*{ii) SII}

\begin{center}
\AxiomC{$S : \xi \rightarrow (\xi \rightarrow \eta)$} 

\AxiomC{$I: \xi$}
\BIC{$\mathbf{SI} : \xi \rightarrow \eta $}

\AxiomC{$ I: \xi$}
\BIC{$\vdash \mathbf{SII} : \eta$}
\DP
\end{center}

\begin{gather*}
\mathbf{S} \equiv  \lambda x. \lambda y. \lambda z. xz(yz) : (\alpha \rightarrow \beta \rightarrow \gamma) \rightarrow (\alpha \rightarrow \beta) \rightarrow \alpha \rightarrow \gamma \\
\mathbf{I} \equiv \lambda w.w : \delta \rightarrow \delta
\end{gather*}

\begin{center}
\AxiomC{$\mathbf{S} : (\alpha \rightarrow \beta \rightarrow \gamma) \rightarrow (\alpha \rightarrow \beta) \rightarrow \alpha \rightarrow \gamma$} 

\AxiomC{$\mathbf{I}: \delta \rightarrow \delta$}
\BIC{$\mathbf{SI} : (\beta \rightarrow \gamma) \rightarrow ((\delta \rightarrow \delta) \rightarrow \beta) \rightarrow (\delta \rightarrow \delta) \rightarrow \gamma$}

\AxiomC{$ I: \delta \rightarrow \delta$}
\BIC{$\vdash \mathbf{SII} : \gamma \rightarrow ((\delta \rightarrow \delta) \rightarrow (\delta \rightarrow \delta)) \rightarrow (\delta \rightarrow \delta) \rightarrow \gamma$}
\DP
\end{center}

With $\xi \equiv \delta \rightarrow \delta \equiv \beta \equiv \alpha; \\
\eta \equiv \gamma \rightarrow ((\delta \rightarrow \delta) \rightarrow (\delta \rightarrow \delta)) \rightarrow (\delta \rightarrow \delta) \rightarrow \gamma  \  $


\subsubsection*{iii) $\lambda xy.y(\lambda z.z(yz))$}

\begin{center}
\AxiomC{$y : (\iota \rightarrow \theta) \rightarrow \iota$}
\AxiomC{$z : \iota \rightarrow \theta$}
\BIC{$yz : \iota$}

\AxiomC{$z : \iota \rightarrow \theta$}
\BIC{$z(yz) : \theta$}

\AxiomC{$z : \iota \rightarrow \theta$}
\BIC{$\lambda z.z(yz) : (\iota \rightarrow \theta) \rightarrow \theta$}
\DP
\end{center}


\begin{center}
\AxiomC{$\lambda z.z(yz) : (\iota \rightarrow \theta) \rightarrow \theta$}
\AxiomC{$y : ((\iota \rightarrow \theta) \rightarrow \theta) \rightarrow \zeta$}
\BIC{$y(\lambda z.z(yz)) : \zeta$}

\AxiomC{$y : ((\iota \rightarrow \theta) \rightarrow \theta) \rightarrow \zeta$}
\BIC{$\lambda y.y(\lambda z.z(yz)) : (((\iota \rightarrow \theta) \rightarrow \theta) \rightarrow \zeta) \rightarrow \zeta$}

\AxiomC{$x : \alpha$}
\BIC{$\vdash \lambda xy. y(\lambda z.z(yz)) : ???$}
\DP
\end{center}


With $(\iota \rightarrow \theta) \rightarrow \iota \equiv _? ((\iota \rightarrow \theta) \rightarrow \theta) \rightarrow \zeta$

Because $\iota \not \equiv \iota \rightarrow \theta$


\subsubsection*{iii bis) $\lambda xy. y(\lambda z.(yz))$}
\begin{center}
\AxiomC{$y :\iota \rightarrow \kappa$}
\AxiomC{$z : \iota$}
\BIC{$yz : \kappa$} %\theta

\AxiomC{$z : \eta$}
\BIC{$\lambda z.yz : \eta \rightarrow \kappa$}

\AxiomC{$y : (\eta \rightarrow \kappa) \rightarrow \epsilon$}
\BIC{$y\lambda z.yz : \epsilon$} 

\AxiomC{$y: \gamma$}
\BIC{$\lambda y.y(\lambda z.yz) : \gamma \rightarrow \epsilon$}

\AxiomC{$ x: \alpha $}
\BIC{$\vdash \lambda xy. y(\lambda z.yz) : \alpha \rightarrow (\gamma \rightarrow \epsilon) $}
\DP
\end{center}

\subsection*{Exercise 5.4}
Find terms $M,N \in \lambda$ such that the following holds in $\lambda \rightarrow$.
\subsubsection*{i) $\vdash M : (\alpha \rightarrow \beta) \rightarrow (\beta \rightarrow \gamma) \rightarrow (\alpha \rightarrow \gamma)$}

\begin{center}
\AxiomC{$c : \gamma$}
\AxiomC{$z : \alpha$}
\BIC{$\lambda z.c : \alpha \rightarrow \gamma$}

\AxiomC{$y : \beta \rightarrow \gamma$}
\BIC{$\lambda yz.c : (\beta \rightarrow \gamma) \rightarrow (\alpha \rightarrow \gamma)$}

\AxiomC{$x : \alpha \rightarrow \beta$}
\BIC{$M \equiv \lambda xyz.c : (\alpha \rightarrow \beta) \rightarrow (\beta \rightarrow \gamma) \rightarrow (\alpha \rightarrow \gamma)$}
\DP
\end{center}
 
\subsubsection*{ii) $\vdash N : (((\alpha \rightarrow \beta) \rightarrow \beta) \rightarrow \beta) \rightarrow (\alpha \rightarrow \beta)$}

\begin{center}
\AxiomC{$b : \beta$}
\AxiomC{$y : \alpha$}
\BIC{$\lambda y.b : \alpha \rightarrow \beta$}

\AxiomC{$x : ((\alpha \rightarrow \beta) \rightarrow \beta) \rightarrow \beta$}
\BIC{$N \equiv \lambda xy.b : (((\alpha \rightarrow \beta) \rightarrow \beta) \rightarrow \beta) \rightarrow (\alpha \rightarrow \beta)$}
\DP
\end{center}

\subsection*{Exercise 5.5}
Find the types in $\lambda 2$ for the terms in the exercises 5.2 and 5.3

\subsubsection*{$\lambda x.xx$}

\begin{center}
\AxiomC{$x : \gamma \rightarrow \beta$} \AxiomC{$ x:  $}
\BIC{$xx : \beta$}

\AxiomC{$ x: \alpha$}
\BIC{$\vdash \lambda x.xx :\ ???$}
\DP
\end{center}

$x$ cannot have simultaneously two different type ``forms": $\gamma \rightarrow \beta$ and $\alpha \equiv \gamma$

\begin{gather*}
\mathbf{KI}(\lambda x.xx) = (\lambda ab.a) (\lambda c.c) (\lambda x.xx) = \\
= (\lambda b.(\lambda c.c))(\lambda x.xx) = \\
= \lambda c.c : \gamma \rightarrow \gamma 
\end{gather*}

\subsection*{$\lambda$2}

\begin{prooftree}
\AxiomC{$\Gamma \vdash M : \forall \alpha . \sigma $}
\UIC{$\Gamma \vdash M \tau : \sigma [\tau/ \alpha]$}
\end{prooftree}

\begin{prooftree}
\AxiomC{$\Gamma \vdash M : \sigma $}
\UIC{$\Gamma \vdash \lambda \alpha . M : \forall \alpha . \sigma$}
\end{prooftree}

\section{Exercises chapter 6 - Barendregt}
\subsection*{Exercise 6.1}
Let $k_n$ be defined by $k_0 \equiv I$ and $k_{n+1} \equiv K(k_n)$. Show that on the $k_n$ the recursive functions can be represented by terms in the $\lambda \delta_C\ calculus$\\
\\
$k_n \equiv K(k_{n-1})$

\begin{gather*}
\mathbf{S^+}\ k_n = k_{n + 1} \\
\mathbf{P^-} k_{n + 1} = k_n \\
\mathbf{Zero}\ k_0 = \mathbf{true} \\
\mathbf{Zero}\ k_{n + 1} = \mathbf{false}
\end{gather*}

\begin{gather*}
\mathbf{S^+}\ = \lambda x. Kx \\
\mathbf{P^-} = \lambda x. K_*x \\
\mathbf{Zero}\ = \lambda n. ((n (\lambda x.K_*)) K) \\
\mathbf{Zero}\ = \lambda n.((\lambda m. (n K)) K) \\
\\
\mathbf{Zero}\ k_0 = \lambda n.((\lambda m. (n K)) K) I = \\
= (\lambda m. (I K)) K = \\
= (\lambda m. K) K = \\
= K \\
\mathbf{Zero}\ k_1 = \lambda n.((\lambda m. (n K)) K) KI = \\
= ((\lambda m. (K K)) K)I = \\
= (\lambda m. (K K)) I = \\
\\
\mathbf{Zero}\ = \lambda n.n T \\
\mathbf{Zero}\ k_0 = (\lambda n.n T) I = \\
= I T = T \\
\mathbf{Zero}\ = (\lambda n.n T)KI = \\
= K T I = T \\
\end{gather*}

\begin{gather*}
k_1 = K(k_0) = KI \\
k_2 = K(k1) = K(KI) \\
k_3 \equiv K(K(KI)) \\
k_4 \equiv K(K(K(KI))) \\
\mathbf{P^-} k_2 = (\lambda x. Kxx) (K(KI)) = \\
= K(K(KI))(K(KI)) = K(KI)
\end{gather*}

\begin{verbatim}
T = Lx.Ly.x 
F = Lx.Ly.y
af = Lx.F
Ln.((n af) T)
\end{verbatim}

\begin{gather*}
Zero = \lambda n.n (\lambda x. false) true \\
Zero\ k_0 = (\lambda n.n (\lambda x. false) true) I = \\
= I (\lambda x. false) true = \\
= (\lambda x. false) true = False \\
\\
Zero\ k_1 = (\lambda n.n (\lambda x. false)\ true) KI = \\
= KI (\lambda x. false)\ true = \\
= I\ true = true \\
\\
Zero\ k_1 = (\lambda n.n (\lambda x. false)\ true) KI = \\
= K((\lambda x. false)\ true) I = \\
= ((\lambda x. false)\ true) = \\
= (\lambda x. false)\ true = false \\
\\
Zero\ k_2 = (\lambda n.n (\lambda x. false)\ true) (K(KI)) = \\
= K(KI)(\lambda x. false)\ true = \\
= KI\ true = \\
= I \\
\\
Zero\ k_3 = (\lambda n.n (\lambda x. false)\ true) (K(K(KI)) = \\
= K(K(KI)) (\lambda x. false)\ true = \\
= K(KI) true = \\
\end{gather*}

























\subsection{Exercise 6.2}
Write down a $\lambda \delta-term\ F$ in the system of Example 6.7 such that 
\begin{center} $Fn \rightarrow n! + n$. \\\end{center}

Factorial: $Fac(n) = if\ (n = 0)\ then\ 1\ else\ (n \times Fac(n-1))$

\clearpage

\begin{verbatim}
(define Fac
  (lambda (n)
     (if (= n 0) 1
           (* n (Fac (- n 1)))
     )
  )
)

(define F
  (lambda (n)
    (+ (Fac n) n)
  )
)
\end{verbatim}

\begin{gather*}
\mathbf{Fac}\ n = \mathbf{ite}\ (\mathbf{equal}\ \mathbf{0}\ n)\ \mathbf{1}\ (\mathbf{times}\ n\ \mathbf{Fac}(\mathbf{minus}\ n\ \mathbf{1}) \\
\mathbf{F}n = \mathbf{plus}\ (\mathbf{Fac}\ n)\ n
\end{gather*}




\section*{Chapter 7 - Barendregt}
\subsection*{Term rewrite systems (TRSs)}
\subsubsection*{Add}

\begin{gather*}
\mathbf{A}(\mathbf{0}, y) \rightarrow y, \\
\mathbf{A}(\mathbf{S}(x), y) \rightarrow \mathbf{S}(\mathbf{A}(x, y))
\end{gather*}

\[
\mathbf{Add} = \lambda xy.\ if\ \mathbf{Zero}\ x\ then\ y\ else\ \mathbf{Succ}(\mathbf{Add}\ (\mathbf{Pred}\ x)y)
\]

\clearpage

\subsubsection*{Mult}

\begin{gather*}
\mathbf{M}(\mathbf{0}, y) \rightarrow \mathbf{0}, \\
\mathbf{M}(x, \mathbf{0}) \rightarrow \mathbf{0}, \\
\mathbf{M}(\mathbf{S}(\mathbf{0}), y) \rightarrow y, \\ 
\mathbf{M}(\mathbf{S}(x), y) \rightarrow \mathbf{A}\ (y, (\mathbf{M} (x, y)))
\end{gather*}

\begin{flalign*}
& \mathbf{Mult} = \lambda xy. if\ \mathbf{Zero}\ x\ 0\ else\ ( & \\
& \hspace{80pt} if\ \mathbf{Zero}\ y\ 0\ else\ ( & \\
& \hspace{95pt} if\ \mathbf{Zero}\ (\mathbf{Pred}\ x)\ then\ y\ else\ \mathbf{Add}\ y\ (\mathbf{Mult} (\mathbf{Pred}\ x)y) & \\
& \hspace{80pt} ) & \\
& \hspace{65pt} ) &
\end{flalign*}






\end{document}